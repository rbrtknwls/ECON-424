\documentclass{article}
\usepackage[margin=1in]{geometry}
\usepackage{amsmath}
\begin{document}
\begin{titlepage}
	\setlength{\parindent}{0pt}
	\large

\vspace*{-2cm}

University of Waterloo \par
Econ 424 \par
\vspace{0.05cm}
Null: 2023-09-12
\vspace{0.2cm}

{\huge Prediction Competition \# 1 \par}
\hrule

\vspace{1cm}
\textbf{Q2)} To determine how much Jimmys expected lifetime will increase, we need to detetermine the difference between his expected income with 12 years of education ($\hat \theta_{12}$) and his expected income with 13 years of education ($\hat \theta_{13}$). In other words, this can be expressed as: \par
\[ \text{ Expected lifetime income change } = \hat \theta_{13} - \hat \theta_{12} \]
By visual observation, we can arrive at $\theta_{12} \approx 133000$ and $\theta_{13} \approx 148000$, plugging this back in gives:
\begin{align*} 
\text{ Expected lifetime income change } &= 148000 - 133000 \\
&=  15000
\end{align*}
So we would expected the expected lifetime income of Jimmy to increase by 15000 if he had another year of education. \par

\vspace{0.7cm}

\textbf{Q3)} Let X = \{$x_1, x_2, .... x_{100}$\}, were $x_i$ represents the expected income of individual point i on the graph. Therefore the average income of the economy can be represented as:
\begin{align*} 
\text{ Average expected income } &= \frac{\sum^{100}_{i=1}x_i}{100} \\
&=  164900
\end{align*}
If everyone had 17 years of income ($\hat \theta_{17}$) then we would expect that the average lifetime income to be:
\begin{align*} 
\text{ Average expected income } &= \frac{\sum^{100}_{i=1}\hat \theta_{17} }{100} \\
&=  213000
\end{align*}
Therfore we would expected that the average lifetime income would increase by 48100 if everyone had 17 years of education. \par

\vspace{0.7cm}

\textbf{Q4)} For the first question, ChatGPT recommends using a linear interpolation to estimate the values of each point. To do this we would find the two closest points and draw a line between then, then on the x axis we would draw a line from the given amount of education and where it interesects with our linear interpolation would be our estimate for the value.\\\\
For the second question, since we dont have access to the data points ChatGPT recommends getting a point estimate for the new position at 13 years of education and finding the difference to the point estimate for 12 years of age. \\\\
Lastly ChatGPT recommends to count up all the points of the graph to calculate the average lifetime income, it then recommends finding a point estimate for income at 17 years of education and then caluclating the difference. 


\end{titlepage}
\end{document}