\documentclass{article}
\usepackage[margin=1in]{geometry}
\usepackage{amsmath,amsfonts,amssymb}
\usepackage{listings}
\usepackage{color}
\usepackage{graphicx}
\usepackage{subfig}
\usepackage{blkarray}
\usepackage{multirow}
\usepackage{float}
\usepackage{caption}
\usepackage{subcaption}
\usepackage{dcolumn}
\usepackage{booktabs}
\usepackage{tikz}
\usetikzlibrary{positioning,shapes,arrows}
\newcolumntype{P}[1]{>{\centering\arraybackslash}p{#1}}
\newcolumntype{M}[1]{D{.}{.}{1.#1}}
\captionsetup[sub]{}

\begin{document}
Let p be the probability that player 1 picks T.
\begin{align*}
\pi_2(L) &= 1*p + 2(1-p) \\
&= p + 2 - 2p \\
&= 2 - 1p \\\\
\pi_2(R) &= 2*p + 1(1-p) \\
&= 2p + 1 -p \\
&= 1 + p \\\\
\end{align*}
Setting them equal gives:
\[ p = \frac{1}{2} \]
Let q be the probability that player 2 picks L.
\begin{align*}
\pi_1(T) &= 0*q + (1-q)*0 \\
&= 0 \\\\
\pi_1(B) &= 2*q + 0*q \\
&= q
\end{align*}
Setting them equal gives:
\[ q = 0? \]



\end{document}